\documentclass[11pt]{article}

\usepackage[utf8]{inputenc}
\usepackage{amsmath, amssymb, amsthm}
\usepackage[inkscapeformat=pdf]{svg}
\usepackage{placeins}
\usepackage{tabularx}
\usepackage{float}
\usepackage{setspace}
\usepackage{hyperref}
\usepackage{enumitem}
\usepackage{parskip}
\usepackage{ffcode}
\usepackage{graphicx}

\usepackage[margin=2cm]{geometry}

\def\code#1{{\texttt{#1}}}

\title{%
  \textbf{DD} \\
  \large Software Engineering Project \\ A.Y. 2022-2023}

\author{Marco Ronzani, Alessandro Sassi}

\date{December 2022}

\begin{document}

\maketitle

\doublespacing
\tableofcontents
\singlespacing

\section{Introduction}
\label{section:introduction}

\subsection{Purpose}

Electric vehicles are starting to grow in number, and their takeover of combustion engines is bound to happen, consequently to support such a thriving trend adequate easy access to charging stations is of utmost importance. In this landscape the goal of eMall is to allow owners of electric vehicles to easily know where charging stations are and carefully plan their charging process according to their schedules at any such station. \\
\\
This document will follow up on the RASD document with a discussion of the architectural design of the system, its main architectural components and their interfaces. It will also cover the plans for implementation and integration, as well as those for testing, with the goal of guiding the development process. \\

\subsection{Scope}

\subsection{Definitions, Acronyms and Abbreviations}

\subsubsection{Definitions}

\begin{table}[H]
    \centering
    %space between text and right/left borders
    \setlength{\tabcolsep}{18pt}
    %Row height multiplier
    \renewcommand{\arraystretch}{1.2}
    \begin{tabularx}{\textwidth}{|>{\centering\hsize=0.4\hsize}X|>{\hsize=1.6\hsize}X|}
        \hline
        \textbf{Term} & \textbf{Definition} \\
        \hline
        Charging Station & Device with a connection to the electric grid which brakes out power to one or more socket(s) for vehicles. Monitoring of each socket's status. \\
        \hline
        Socket & One of the charging outlets available at a CS where a vehicle connects, its type determines the vehicles that can connect. \\
        \hline
        Charging Session & The process in which a User performs a recharge of their vehicle at a specific socket. \\
        \hline
        Energy source of a Charging Station & Batteries or the DSO currently assigned to the CS, whichever the CS is currently drawing its power from. \\
        \hline
        Charging Station External Status & Number of charging sockets available, their type such as slow/fast/rapid, their cost, and, if all sockets of a certain type are occupied, the estimated amount of time until the first socket of that type is freed. \\
        \hline
        Charging Station Internal Status & Amount of energy available in the batteries, if any, number of vehicles being charged and, for each charging vehicle, amount of power absorbed and time left to the end of the recharge. \\
        \hline
        Energy Source & Method of energy production that results in a known fraction of the energy supplied to an endpoint in the electric grip. \\
        \hline
        User-price & Cost of a recharge that is shown to the User when they inspect a CS and is what they are charged for. It is set by the CPO/CPMS on a per-CS basis.\\
        \hline
        Nominal-price & Cost of a recharge at a CS without any discount or offer applied, it is set by the CPO/CPMS. A user-price $<$ nominal-price implies an ongoing special offer. \\
        \hline
        Energy source management policy OR Battery usage policy & A per-CS policy given them by the CPMS to allow them to dynamically decide whether to acquire energy from their assigned DSO or from their batteries and when to charge their batteries with energy from their DSO. \\
        \hline
        Charge Point Management System's policy for “Automatic Mode” & The global policy used by the CPMS to operate autonomously, its mainly built around thresholds and weights to allow the CPMS to decide prices and battery usage policies for its CSs. \\
        \hline
    \end{tabularx}
    \label{tab:definitions}
\end{table}

\subsubsection{Acronyms}

\begin{table}[H]
    \centering
    %space between text and right/left borders
    \setlength{\tabcolsep}{18pt}
    %Row height multiplier
    \renewcommand{\arraystretch}{1.2}
    \begin{tabularx}{\textwidth}{|>{\centering\hsize=0.3\hsize}X|>{\hsize=1.7\hsize}X|}
        \hline
        \textbf{Acronym} & \textbf{Full Name} \\
        \hline
        eMall & Electric Mobility for All \\
        \hline
        eMSP & Electric Mobility Service Provider \\
        \hline
        CPO & Charging Point Operator \\
        \hline
        CPMS & Charge Point Management System \\
        \hline
        CS & Charging Station \\
        \hline
        DSO & Distribution System Operator \\
        \hline
        STB & System-To-Be \\
        \hline
        OS & Operating system \\
        \hline
    \end{tabularx}
    \label{tab:acronyms}
\end{table}

\subsection{Revision History}

\subsection{Reference Documents}

\subsection{Document Structure}

\section{Architectural Design}

%For now considering: Express (NodeJs) + MySQL/Postgre

%4-tier architecture (places where layers are deployed) both for eMSP and CPMS:
% >Presentation/Client
% >Web Server
% >App Server
% >DB Server

%Architettura monolitica, vecchio stile
%Giustificazione: la abbiamo progettata così che sia facilmente divisibile in microservice qualora il carico sul sisteme richedesse la distribuzione delle funzionalità su più macchine. Inoltre i benefici alla sicurezza di avera un sistema monolitico non sono trascurabili.

%4-layer to map over the physical entities (not 1:1 necessarily)
%http://www.hanselman.com/blog/a-reminder-on-threemulti-tierlayer-architecturedesign-brought-to-you-by-my-late-night-frustrations
% >Presentation
% >Business logic
% >Data Access
% >Data Store

%Server 1: eMSP + DB
%Server 2: CPMS + DB

%CS client del CPMS

\subsection{Overview}

The architecture of the system is organized in \textbf{four layers} distributed over \textbf{four tiers} in a thin client fashion, this results in a monolithic structure that benefits security and maintenance. To be noted however that the modular design of the architecture is already predisposed to allow an easy transitions to microservices when, if ever, the load imposed on the system will require the distribution of its functionalities over many machines. \\
\\
OVERVIEW DIAGRAM HERE

\begin{itemize}
    \item \textbf{eMSP Web Application} \\
        A Web Application that runs within the User’s web browser and that gives them access to the eMall services. All the resources it needs get downloaded to the User's browser from the eMSP Web Server as the the device reaches it through HTTP requests.
    \item \textbf{eMSP Web Server} \\
        Back-end component that interfaces the User's web browser with the other application services in the eMSP's back-end. It handles HTTP requests arriving from user-devices by routing them to the corresponding components that can respond to them.
    \item \textbf{eMSP App Server} \\
        The principal back-end component for the eMSP, it contains all the modules that together offer all functionalities available via the eMSP. Requests are routed from the Web Server to the module capable handling them, the response is than sent from the App Server through the Web Server to the User's device. The App Server is also capable of accessing the eMSP DataBase and the external services needed for its functions.
    \item \textbf{eMSP DB} \\
        Component dedicated to data storage for the eMSP, it stores for instance the Users' credentials and bookings, to ensure its safety it can only be accessed by the eMSP's App Server.
    \item \textbf{CPMS Web Application} \\
        A Web Application meant for the desktop devices available to CPO Employees, it runs withing the browser and allows its users to access the management functions of the CPMS. It dialogues with the CPMS Web Server through the HTTP protocol.
    \item \textbf{CPMS Web Server} \\
        Back-end component that interfaces the CPO Employees' web browser and the back-ends of the eMSP with the application services in the CPMS's back-end. It handles HTTP requests arriving from the outside by routing them to the corresponding components that can respond to them and allows tunneling of WebSocket connections from CS to the back-end.
    \item \textbf{CPMS App Server} \\
        The principal back-end component for the CPMS, it contains all the modules that together offer all functionalities available to the CPO and eMSPs, as well as being the endpoint where CS connect to in order to be managed. Requests and connections are routed from the Web Server to the module capable handling them, in the fist case the response is than sent from the App Server through the Web Server to the requesting device, in the second case the connection is handled and kept alive by its target module. The App Server is also capable of accessing the CPMS DataBase and the external services needed for its functions.
    \item \textbf{CPMS DB} \\
        Component dedicated to data storage for the CPMS, it stores for instance the credentials for CPO Employees as well as a local copy of the CSs' configurations, to ensure its safety it can only be accessed by the CPMS's App Server.
    \item \textbf{CS} \\
        External entity w.r.t. the STB, it is configured to reach out to the CPMS App Server via a WebSocket connection that allows the CPMS to manage and monitor it.
    \item \textbf{DSO} \\
        External entity w.r.t. the STB, it exposes a web API that allows CPMSs to acquire information regarding its price and energy mix.
    \item \textbf{Maps Provider} \\
        External entity w.r.t. the STB, offers to clients up to date maps of any requested area, as well as related information such as satellite images.
\end{itemize}

\subsection{Component View}

COMPONENT DIAGRAM HERE

\subsubsection{eMSP Web Application Component}

The client devices access via a web browser the responsive webapp.

\subsubsection{eMSP Web Server Component}

Contain a Reverse Proxy component that ... 

\subsubsection{eMSP Application Server Component}

\begin{itemize}
    \item \textbf{Authentication} \\
        aaa
    \item \textbf{Login} \\
        aaa
    \item \textbf{Register} \\
        aaa
    \item \textbf{SearchCS} \\
        aaa
    \item \textbf{BookingsManager} \\
        aaa
    \item \textbf{RechargeManager} \\
        aaa
    \item \textbf{PaymentModule} \\
        aaa
    \item \textbf{DataAccess} \\
        aaa
\end{itemize}

\subsubsection{CPMS Web Application Component}

The CPO's Employees devices via a web browser access the CPMS management web pages.

\subsubsection{CPMS Web Server Component}

Contain a Reverse Proxy component that ... for both HTTP requests and WebSockets from the CS ...

\subsubsection{CPMS Application Server Component}

\begin{itemize}
    \item \textbf{Authentication} \\
        aaa
    \item \textbf{Login} \\
        aaa
    \item \textbf{CSBatteryPolicyManager} \\
        aaa
    \item \textbf{CSDSOManager} \\
        aaa
    \item \textbf{CSPricesManager} \\
        aaa
    \item \textbf{AutomaticModeManager} \\
        aaa
    \item \textbf{CSList} \\
        aaa
    \item \textbf{RechargeManager} \\
        aaa
    \item \textbf{DSOInformation} \\
        aaa
    \item \textbf{DataAccess} \\
        aaa
\end{itemize}

\subsubsection{eMSP Data Components}

%E-R diagram

\subsubsection{CPMS Data Components}

%E-R diagram

\subsubsection{External Systems}

\begin{itemize}
    \item \textbf{DSO} \\
        aaa
    \item \textbf{Payment Provider} \\
        aaa
    \item \textbf{CSBatteryPolicyManager} \\
        aaa
\end{itemize}

%DSO
%Payment
%Maps

\subsection{Deployment View}

\subsection{Runtime View}

\subsection{Component Interfaces}

\subsection{Selected Architectural Styles and Patterns}

\subsection{Other Design Decision}

\section{User Interface Design}

\section{Requirements Traceability}

\section{Implementation, Integration and Test Plan}

%Bottom-up

\section{Effort Spent}

\subsection{Ronzani Marco - mat: 224578}

\begin{tabular}{|l|l|}
    \hline
    \textbf{Task} & \textbf{Time spent} \\
    \hline
    Introduction & $\infty h$ \\
    \hline
    Architectural Design & $\infty h$ \\
    \hline
    User Interface Design & $\infty h$ \\
    \hline
    Requirements Traceability & $\infty h$ \\
    \hline
    Implementation, Integration and Test Plan & $\infty h$ \\
    \hline
    Other & $\infty h$ \\
    \hline
    \hline
    Total & $5*\infty h$ \\
    \hline
\end{tabular}

\subsection{Sassi Alessandro - mat: ...}

\begin{tabular}{|l|l|}
    \hline
    \textbf{Task} & \textbf{Time spent} \\
    \hline
    Introduction & $\infty h$ \\
    \hline
    Architectural Design & $\infty h$ \\
    \hline
    User Interface Design & $\infty h$ \\
    \hline
    Requirements Traceability & $\infty h$ \\
    \hline
    Implementation, Integration and Test Plan & $\infty h$ \\
    \hline
    Other & $\infty h$ \\
    \hline
    \hline
    Total & $5*\infty h$ \\
    \hline
\end{tabular}

\newpage

\section{References}

\end{document}
