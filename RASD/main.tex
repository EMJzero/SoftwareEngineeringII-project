\documentclass[11pt]{article}

\usepackage[utf8]{inputenc}
\usepackage{amsmath, amssymb, amsthm}
\usepackage[inkscapeformat=pdf]{svg}
\usepackage{placeins}
\usepackage{tabularx}
\usepackage{float}
\usepackage{setspace}
\usepackage{hyperref}

\usepackage[margin=2cm]{geometry}

\title{%
  \textbf{RASD} \\
  \large Software Engineering Project \\ A.Y. 2022-2023}

\author{Marco Ronzani, Alessandro Sassi}

\date{November 2022}

\begin{document}

\maketitle

\doublespacing
\tableofcontents
\singlespacing

\section{Introduction}

\subsection{Purpose}

Electric vehicles are starting to grow in number, and their takeover of combustion engines is bound to happen, consequently to support such a thriving trend adequate easy access to charging stations is of utmost importance. In this landscape the goal of eMall is to allow owners of electric vehicles to easily know where charging stations are and carefully plan their charging process according to their schedules at any such station.\\
\\
This document will discuss goals and requirements that regard the system necessary to make eMall a reality, with the purpose of guiding the development process.

%Description of the System To Be:
%It is composed of the eMSP and one or more CPMSs ... description
%=> CPMS non va condiserato come actor perchè parte del STB

%Actor: User, DSO, CPO

%Consideration:
%CPMSs are not necessarily part of eMall, they might be entirely separated and can be reasonably assumed to exist independently.

%A CS has associated a DSO (one exactly at every instant) and is "plugged" thanks to it in the electric grid. The management of batteries is also handled per-CS.

%The user is charged a price that is exactly the energy he used times the original price he booked the CS for, hence eventual changes operated by the CPMS during the process that alter the cost of electricity do not affect the user's charge, simply the difference kept or given by the CPO/CPMS.

%A user should be able to show up whenever at a free socket and charge by "booking it on the fly" for the time window the socket is free!

%Machine: eMSPs + CPMSs
%World: Users + CS (charging stations) + Sockets + DSOs + CPOs

\subsubsection{Goals}

\begin{table}[H]
    \centering
    %space between text and right/left borders
    \setlength{\tabcolsep}{18pt}
    %Row height multiplier
    \renewcommand{\arraystretch}{1.2}
    \begin{tabularx}{\textwidth}{|>{\centering\hsize=0.3\hsize}X|>{\hsize=1.7\hsize}X|}
        \hline
        \textbf{Goal} & \textbf{Description} \\
        \hline
        G1 & Allow Users to see the list of available Charging Stations \\
        \hline
        G2 & Allow Users to see the cost of a charge \\
        \hline
        G3 & Allow Users to book charging time slots \\
        \hline
        G4 & Allow Users to monitor, manage and pay for the ongoing charging session \\
        \hline
        G5 & Allow Charge Point Operators to manage the provider each Charging Station uses to source energy \\
        \hline
    \end{tabularx}
    \label{tab:goals}
\end{table}

\subsection{Definitions, Acronyms and Abbreviations}

\subsubsection{Definitions}

\begin{table}[H]
    \centering
    %space between text and right/left borders
    \setlength{\tabcolsep}{18pt}
    %Row height multiplier
    \renewcommand{\arraystretch}{1.2}
    \begin{tabularx}{\textwidth}{|>{\centering\hsize=0.3\hsize}X|>{\hsize=1.7\hsize}X|}
        \hline
        \textbf{Term} & \textbf{Definition} \\
        \hline
        ... & Should we define here what a CPMS, CPO, eMSP, etc. are and should more or less do? \\
        \hline
        Charging Station & Device with a connection to the electric grid which brakes out power to one or more socket(s) for vehicles. Monitoring of each socket's status. \\
        \hline
        Charging Session & The process in which a User performs a charge of their vehicle. \\
        \hline
        Energy Source & Method of energy production that results in a known fraction of the energy supplied to an endpoint in the electric grip. \\
        \hline
    \end{tabularx}
    \label{tab:definitions}
\end{table}

\subsubsection{Acronyms}

\begin{table}[H]
    \centering
    %space between text and right/left borders
    \setlength{\tabcolsep}{18pt}
    %Row height multiplier
    \renewcommand{\arraystretch}{1.2}
    \begin{tabularx}{\textwidth}{|>{\centering\hsize=0.3\hsize}X|>{\hsize=1.7\hsize}X|}
        \hline
        \textbf{Acronym} & \textbf{Full Name} \\
        \hline
        eMall & Electric Mobility for All \\
        \hline
        eMSP & Electric Mobility Service Provider \\
        \hline
        CPO & Charging Point Operator \\
        \hline
        CPMS & Charge Point Management System \\
        \hline
        CS & Charging Station \\
        \hline
        DSO & Distribution System Operator \\
        \hline
        STB & System-To-Be \\
        \hline
    \end{tabularx}
    \label{tab:acronyms}
\end{table}

\subsection{Scope}

This RASD document takes into consideration the requirements and specifications of the eMSP platform “eMall”, together with its interaction with one or more CPMSs. The stakeholders considered are the End Users who interact with the “eMall” platform, CPOs owning the respective CSs and CPMSs, and DSOs offering their services to the aforementioned parties.

\subsubsection{World Phenomena}

\begin{table}[H]
    \centering
    %space between text and right/left borders
    \setlength{\tabcolsep}{18pt}
    %Row height multiplier
    \renewcommand{\arraystretch}{1.2}
    \begin{tabularx}{\textwidth}{|>{\centering\hsize=0.3\hsize}X|>{\hsize=1.7\hsize}X|}
        \hline
        \textbf{Phenomena} & \textbf{Description} \\
        \hline
        WP1 & Electric vehicles connect to a CS socket \\
        \hline
        WP2 & CPOs add/remove available CSs \\
        \hline
        %DOMAIN ASSUMPTION: If you change sockets you change tho whole CS!
        WP3 & CPOs add/remove batteries from existing CSs \\
        \hline
        WP4 & CPOs decide the policy with which to assign DSOs to CSs \\
        \hline
        WP5 & DSOs set the price and/or the mix of sources for the electricity they provide \\
        \hline
        WP6 & CPOs pay DSOs for the consumed electricity \\
        \hline
    \end{tabularx}
    \label{tab:world_phenomena}
\end{table}

\subsubsection{Shared Phenomena}

In order to represent more clearly whether a phenomenon is world or machine controlled or observed, we define a list of acronyms used only for the scope of the Shared Phenomena definition. These acronyms will also be reported in the Shared Phenomena table, so that for each entry the controller/observer party can easily be identified. 

\begin{table}[H]
    \centering
    %space between text and right/left borders
    \setlength{\tabcolsep}{18pt}
    %Row height multiplier
    \renewcommand{\arraystretch}{1.2}
    \begin{tabularx}{\textwidth}{|>{\centering\hsize=0.3\hsize}X|>{\hsize=1.7\hsize}X|}
        \hline
        \textbf{Type} & \textbf{Explanation} \\
        \hline
        MO & Machine Observed Phenomenon, the STB takes notice of the event \\
        \hline
        MC & Machine Controlled Phenomenon, the STB causes the event \\
        \hline
        WO & World Controlled Phenomenon, an/some external agent(s) notices the event \\
        \hline
        WC & World Controlled Phenomenon, an/some external agent(s) causes the event \\
        \hline
    \end{tabularx}
    \label{tab:shared_phenomena_header}
\end{table}

\begin{table}[H]
    \centering
    %space between text and right/left borders
    \setlength{\tabcolsep}{18pt}
    %Row height multiplier
    \renewcommand{\arraystretch}{1.2}
    \begin{tabularx}{\textwidth}{|>{\centering\hsize=0.3\hsize}X|>{\centering\hsize=0.3\hsize}X|>{\hsize=1.4\hsize}X|}
        \hline
        \textbf{Phenomena} & \textbf{Type} & \textbf{Description} \\
        \hline
        SP1 & WC MO & The User views the available charging stations and their information \\
        \hline
        SP2 & WC MO & Users book a charging session to a CS \\
        \hline
        SP3 & WC MO & Users start their vehicle's charging session on a CS \\
        \hline
        SP4 & MC WO & The eMSP notifies users of the completion of their vehicle's charging session \\
        \hline
        SP5 & WC MO & Users prematurely terminate their ongoing charging session \\
        \hline
        SP6 & WC MO & Users pay for the finished charging session \\
        \hline
        %Since CPOs need those to manually manage CSs!!
        SP7 & MC WO & CPMSs publish the location and external status information regarding their managed CSs \\
        \hline
        %There are two energy prices, the one the DSOs charge the CPOs for and the one the CPOs charge the Users for!
        SP8 & WC MO & CPOs update the DSO associated to a CS (this determines how much a CPO is charged for the electricity a CS provides) \\
        \hline
        SP9 & WC MO & CPOs update the "user-charged" energy price and set special offers for a CS \\
        \hline
        SP10 & WC MO & CPOs perform an addition/removal of CSs \\
        \hline
        SP11 & WC MO & CPOs update batteries information and batteries usage policies for their CSs \\
        \hline
        SP12 & MC WO & The eMSP transfers the amount due to CPOs for performed charges \\
        \hline
    \end{tabularx}
    \label{tab:shared_phenomena}
\end{table}

\subsection{Revision History}

\begin{enumerate}
    \item[v0.1] First draft of the document.
\end{enumerate}

\subsection{Reference Documents}

\begin{enumerate}
    \item The provided document describing the project: \textit{Assignment RDD AY 2022-2023-v3}.
    \item \href{https://www.platformelectromobility.eu/2022/05/17/ev-charging-how-to-tap-in-the-grid-smartly/}{Platform for Electromobility. EV Charging: How to tap in the grid smartly?}
    \item \href{https://mobilityintegrationsymposium.org/wp-content/uploads/sites/10/2018/11/4A_3_Emob18_024_paper_Filipe_Campos.pdf}{F. Campos, L. Marques, and K. Kotsalos, Electric Vehicle CPMS and Secondary Substation Management}.
\end{enumerate}

\subsection{Document Structure}

\section{Overall Description}

\subsection{Product Perspective}

\subsubsection{Scenarios}
\label{subsubsec:scenarios}

\begin{description}
    \item [1. User looking for charging stations] \hfill \\
        \textit{Actors: user} \\
        A person, Bob, needs to know the location and/or pricing of CSs in a certain are of his interest. To obtain such data, Bob has reached the eMall services and has selected the charging stations map options. Now Bob can search for CSs and is presented with multiple filters to control: \\
        \begin{itemize}
            \item His area of interest
            \item Charge price ranges
            \item Connector types available
            \item Charging speeds available
        \end{itemize}
        By manipulating the filters Bob can restrict or widen the scope of his search. Results of the search are displayed in respect of the filters and Bob can see the details of every result such as:
        \begin{itemize}
            \item Location
            \item Price
            \item Offers
            \item Booking schedule
            \item Available connectors
            \item Available charging speeds
        \end{itemize}
    \item [2. New user registration] \hfill \\
        \textit{Actors: user} \\
        A person, Bob, who discovered eMall has decided to register into the platform to use its charge booking and charge management system. To this end Bob reaches the eMall services and chooses to “sign up”, after that Bob fills in the relevant information and his eMall account is created. Now Bob can proceed to login.
    \item [3. User login] \hfill \\
        \textit{Actors: user} \\
        An already registered user, Bob, reaches the eMall services and intends to either book a charging spot or to manage a currently ongoing charge booked to his name, hence Bob chooses to “login” and enters its credentials. Assuming Bob inserted valid credentials he now has a valid session within the eMall services. If Bob failed in setting valid credentials he can try again.
    \item [4. User booking a charge] \hfill \\
        \textit{Actors: user} \\
        A user Bob has already performed the login and now intends to book a charge. Bob first performs a search for the CS to reserve for his charge (see “user looking for charging stations” scenario) and once he finds a suitable and available one chooses “book charge” for that station. Bob is then presented with the choice for the time slot for his charge and once he chooses his reservation is completed.
    \item [5. User managing a charge / user performing a charge (start, monitor and pay)] \hfill \\
        \textit{Actors: user} \\
        The user Bob has previously booked a charge at a certain CS for a certain time slot. Bob shows up at the CS during his reserved time slot and proceeds to connect his vehicle properly. Once the vehicle is connected Bob reaches the eMall services to start the charging process. After the process is started Bob has the option to interrupt it at any time, otherwise the charging procedure automatically halts as the CS noticed the vehicle’s battery to be full and notifying Bob as that happens, either way Bob is then presented by eMall with the total cost of his charge and a payment module which Bob must fill to complete the payment process.
    \item [6. CPO wants information on DSOs’ energy prices and mix of sources] \hfill \\
        \textit{Actors: CPO} \\
        The CPO Xcorp, or better its employee Mary which authorizes herself with Xcorp's CPMS, wants all information regarding the energy market to evaluate their choice of a DSO. To do this Xcorp uses its CPMS’s functionality to gather such information from all the DSOs known to it, hence the CPMS proceeds to recover this information from the various DSOs and once it has finished it presents the result to Xcorp.
    \item [7. CPO chooses energy sources and battery usage policies for a CS] \hfill \\
        \textit{Actors: CPO} \\
        The CPO Xcorp might want to change which DSO supplies its different CSs and how the CSs manage their batteries, if they have any, in order to offer better prices, to cut its expenses or take advantage of better energy mixes provided by other DSO. To this end Xcorp instructs one of its employees, Mary, to authenticates with Xcorp's CPMS. Afterwards Mary selects the CS(s) which will be subject to changes and can choose to assign that CS(s) a new DSO among those the CPMS is aware of or to change the battery usage policy for that CS(s).
    \item [8. CPO toggles the CPMS automatic mode] \hfill \\
        \textit{Actors: CPO} \\
        A CPO employee, Mary, in charge of administering the CPMS system wants to change its operating mode from Manual to Automatic. To do this, she logs in to the CPMS and toggles “Automatic Mode" on. Once the system has finished changing the operating mode, she loses control over manual overrides, and is prompted to enable "Manual Mode" again in order to change those settings. After completing the action, Mary successfully logs out of the system. \hfill \\
\end{description}

\subsubsection{Class Diagram}

UML class diagram showing the entities of interest for the system and their relationships. \\
\\
A notably missing entity is \textit{vehicle}, which was deemed not necessary, since its charge status is modelled within the socket it is connected to and the user might not necessarily be the owner of the vehicle he charges (ex: rented vehicle), so binding the two might have been counterproductive. \\
\\
\textit{Unregistered User} is not present and modelled as a flag of User. !!SHOULD WE MODEL IT? Since its not connected to any booking, a significant difference with User...!! 

\subsubsection{State Diagrams}

Those following are representations of the most important interactions that involve the STB, they mostly correspond to the different \hyperref[subsubsec:scenarios]{scenarios}.

\subsection{Product Functions}
\label{subsec:prodfunctions}

!!THE MOST IMPORTANT REQUIREMENTS!!

(- showing CSs (prices etc)
- booking charge
- terminating charge
- notifying charge completion
- paying)

(- CS status
- start charging vehicle
- monitor charging process (to determine a full charge)
- swap DSO for a CS
- change CS batteries behaviour)

\subsection{User Characteristics}

\begin{description}
    \item [1. Unregistered User] \hfill \\
        A User which just accesses the list of available CS without being able to book charges, he has the option to register and become a \textit{Logged-In User} at any time.
    \item [2. Logged-In User] \hfill \\
        A User with a registered account within eMall which has logged in via their credentials, can still access the CSs list and can also book charge. With their bookings he can then manage their charging sessions one he arrives at the CS.
    \item [3. CPO] \hfill \\
        By CPO at large we mean a company whose employees have access to the company's CPMS, hence the actual human beings interacting with the STB will be CPO employees with credentials that allow them to authenticate with the CPMS. The tasks an employee might perform are acquiring DSOs' information, assigning a DSO to a CS owned by their company, changing the battery management policy for any CS controlled by the CPMS or adding/updating CS managed by the CPMS. \\
        The reason for CPOs being identified as a users instead of their employees is that the decisions justifying the operations performed by the employees cannot be taken by the employees, they are just the executors, decisions are taken by their superiors, hence we can view of the entire company as a user to avoid being too specific about who takes which decision and who interacts with the STB. \\
        Furthermore a CPO is likely to use its information system to interface with the CPMS, hence identifying the company as a user also prevents distintions between direct CPMS access or access via the information system.
\end{description}

\subsection{Assumptions, Dependencies and Constraints}

\subsubsection{Domain Assumption}

\begin{table}[H]
    \centering
    %space between text and right/left borders
    \setlength{\tabcolsep}{18pt}
    %Row height multiplier
    \renewcommand{\arraystretch}{1.2}
    \begin{tabularx}{\textwidth}{|>{\centering\hsize=0.3\hsize}X|>{\hsize=1.7\hsize}X|}
        \hline
        \textbf{Assumption} & \textbf{Description} \\
        \hline
        D1 & To change the sockets of a CS, the whole CS needs to be replaced \\
        \hline
        D2 & The only way to charge at a CS is by booking it via the eMSP \\
        \hline
        D3 & Each socket of a CS is free at any given time unless it's booked for that time slot \\
        \hline
        D4 & The only user using a socket of a CS during a given time slot is the one who booked said socket \\
        \hline
        D5 & A DSO's energy price is constant for all the locations where said DSO serves electricity \\
        \hline
        D6 & A DSO's energy mix is constant for all the locations where said DSO serves electricity \\
        \hline
        D7 & The DSO association can be controlled on a per-CS basis and there is enough information provided by the DSOs to the CPMS to allow the latter to set a price for a charge at a CS and keep it constant during the whole process, while preventing that a DSO's price change causes a loss \\
        \hline
        D8 & CPOs pay DSOs for the used electricity via methods outside the scope of this document (Ex: direct bank transfer) \\
        \hline
        D9 & CSs are aware of whether a vehicle is connected or not, and if it is connected they know its charging state \\
        \hline
    \end{tabularx}
    \label{tab:domain_assumptions}
\end{table}

\section{Specific Requirements}

\subsection{External Interface Requirements}

\subsubsection{User Interfaces}

The interface eMall will use is a web app, through which its Users will have available all \hyperref[subsec:prodfunctions]{product functions}. Such web app will need to be usable both on desktop and mobile devices, since a mobile device is what users will most likely have available to them while at the CS. \\
\\
Similarly, CPMSs will offer their functionality to CPOs via a web portal, here there is no need to support anything other than a desktop device, since CPMSs are intended to be managed by employees in a company environment.

\subsubsection{Hardware Interfaces}

Users will require a device equipped with a web browser and an internet connection to access the eMall website. \\
\\
CSs will need to be equipped with adequate hardware to monitor each socket and control circuitry for their batteries, if there are any connected to them. Furthermore CSs need to communicate with their respective CPMSs, hence a small computer within each CS capable of collecting sockets and battery(ies) data and interface with the internet (via 4G or a wired connection) is necessary.

\subsubsection{Software Interfaces}
%The information exchanged between different SW (external?) entities

Map service and payment system...

\subsubsection{Communication Interfaces}
%The medium over which information is exchanged

% CPMS - eMSP
% CPMS - CSs
% CPMS - DSO
% eMSP - payment (user pays the eMSP for a charge)
% ci deve essere anche un modo per l'eMSP di pagare il CPO...


The main functionalities of the STB require cooperation of multiple components sometimes owned by different parties and not deployed on machines in the same local network, hence their interaction occurs via adequate web APIs. The interactions requiring such interfaces are the following:

\begin{description}
    \item [1. eMSP acquiring CS information from CPMSs] \hfill \\
        Pew Pew Bla Bla...
    \item [1. eMSP performing payment via external provider] \hfill \\
        Pew Pew Bla Bla...
    \item [1. eMSP monitoring a charge and managing a charge through a CPMS] \hfill \\
        Pew Pew Bla Bla...
    \item [1. CPMS monitoring and controlling its CSs] \hfill \\
        Pew Pew Bla Bla...
    \item [1. CPMS acquiring DSOs price and energy mix information] \hfill \\
        Pew Pew Bla Bla...
\end{description}

Its assumed that a CS is billed for its used electricity via a in loco electricity meter monitored by the DSO currently assigned to the CS by the CPMS

\subsection{Functional Requirements}

\subsubsection{Requirement assertions}

\begin{table}[H]
    \centering
    %space between text and right/left borders
    \setlength{\tabcolsep}{18pt}
    %Row height multiplier
    \renewcommand{\arraystretch}{1.2}
    \begin{tabularx}{\textwidth}{|>{\centering\hsize=0.1\hsize}X|>{\hsize=1.9\hsize}X|}
        \hline
        \textbf{Id} & \textbf{Requirement} \\
        \hline
        R1 & The eMSP shall allow a registered user to login \\
        \hline
        R1 & The eMSP shall allow an unregistered user to register on the platform \\
        \hline
        %Maybe specialize for each error
        R1 & The eMSP should report to the user any errors that arise \\
        \hline
        R1 & The eMSP should report to the user any successful action performed \\
        \hline
        R1 & The eMSP shall allow a logged-in user to view charging stations nearby \\
        \hline
        R1 & The eMSP shall allow a logged-in user to know the cost of a charge at a CS \\
        \hline
        R1 & The eMSP shall allow a logged-in user to know about ongoing special offers present at a CS \\
        \hline
        R1 & The eMSP shall allow a logged-in user to book a charge at a free socket of a CS for a given time slot \\
        \hline
        R1 & The eMSP shall allow a logged-in user to cancel any of its booked charges \\
        \hline  
        R1 & The eMSP shall not allow a user to book a charge at a socket of a CS which has already been booked for the chosen time slot \\
        %IMPLIES ONLY A CHARGE ONGOING AT ANY GIVEN TIME
        \hline
        R1 & The eMSP shall not allow a user to book multiple charges with overlapping time slots \\
        \hline
        R1 & The eMSP shall allow a logged-in user to start the charging process for one of their booked charges \\
        \hline
        R1 & The eMSP shall not allow a user without a booking to start the charging process \\
        \hline
        R1 & The eMSP shall not allow a user to start the charging process as long as the socket is vacant \\
        \hline
        R1 & The eMSP should notify a logged-in user when their ongoing charging processes is complete \\
        \hline
        R1 & The eMSP should notify a logged-in user when their ongoing charging process is terminated due to the expiration of its booked time slot \\
        \hline
        R1 & The eMSP shall allow a logged-in user to terminate their charging process earlier \\
        \hline
        R1 & The eMSP shall allow a logged-in user to pay for charging process as soon as it finishes \\
        \hline
        R1 & The eMSP shall charge a user for a fee if they do not start a charge during their booking's time slot \\
        \hline
        R1 & The eMSP shall be able to offer its functionalities to multiple users at once \\
        \hline
        % The eMSP shall be able to pay CPOs for the amount due for charges performed through them since the last payment
    \end{tabularx}
    \label{tab:requirements}
\end{table}

\subsubsection{Mapping on goals}

\begin{table}[H]
    \centering
    %space between text and right/left borders
    \setlength{\tabcolsep}{18pt}
    %Row height multiplier
    \renewcommand{\arraystretch}{1.2}
    \begin{tabularx}{\textwidth}{|>{\hsize=0.4\hsize}X|>{\hsize=1\hsize}X|>{\hsize=1.6\hsize}X|}
        \hline
        \textbf{Goal} & \textbf{Domain assumption} & \textbf{Requirement} \\
        \hline
        G0 & D0 & R0 \\
        \hline
        ... & ... & ... \\
        \hline
    \end{tabularx}
    \label{tab:requirements}
\end{table}

\subsubsection{Use cases}
%MAYBE BEFORE REQUIREMENTS

1 to 1 with scenarios?

\subsubsection{Use case diagrams}

\subsubsection{Sequence diagrams}

\subsubsection{Mapping on requirements}

\begin{table}[H]
    \centering
    %space between text and right/left borders
    \setlength{\tabcolsep}{18pt}
    %Row height multiplier
    \renewcommand{\arraystretch}{1.2}
    \begin{tabularx}{\textwidth}{|>{\hsize=1.2\hsize}X|>{\hsize=0.8\hsize}X|}
        \hline
        \textbf{Use case} & \textbf{Requirements} \\
        \hline
        ... & ... \\
        \hline
    \end{tabularx}
    \label{tab:requirements}
\end{table}

\subsection{Performance Requirements}

\subsection{Design Constraints}

\subsubsection{Standards compliance}

\subsubsection{Hardware limitations}

\subsubsection{Any other constraint}

\subsection{Software System Attributes}

\subsubsection{Reliability}

\subsubsection{Availability}

\subsubsection{Security}

Prevent an asshole to create 1 trillion accounts and occupy all of a city worth of CS, hence require credit card info on registration!

\subsubsection{Maintainability}

\subsubsection{Portability}

\section{Formal Analysis Using Alloy}

\section{Effort Spent}

\section{References}

\end{document}